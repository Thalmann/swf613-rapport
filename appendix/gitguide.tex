\chapter{Git guide for Windows}\label{gitguide}
The following section describes the setup of Git on a Windows system.
This guide was written by the Git specialist and is included for additional reference on the use of Git.
The guide was originally written in danish and has not been translated for this report, as it is not an essential part of this report.
\section{Installation af Git}
Git hentes på følgende url: \url{http://git-scm.com/download/win}.

Under installationen kan man vælge at installere Git Bash som terminal.
Det vil jeg personligt ikke anbefale, da der findes en extension til PowerShell der laver \textit{magi} og gør verden (på Windows) til et bedre sted.
Opsætning af dette er forklaret i \ref{gitguide:powershell} \nameref{gitguide:powershell}.
Man kan selvfølgelig også vælge at benytte en GUI til Git, omend det svækker mængden af kontrol man som bruger har over Git.

\paragraph{Test af Git installation}
Start den terminal du har mest lyst til at anvende (cmd.exe kan til nød anvendes her).
Tjek herefter om git er blevet korrekt installeret og at git er tilføjet til din path ved at køre:
\begin{lstlisting}
git --version
\end{lstlisting}
Terminalen skulle gerne svare med din version af git (fx \texttt{git version 1.8.3.msysgit.0}).
Hvis det ikke sker skal du tjekke at stien til git.exe (fx C:\textbackslash{}Program Files (x86)\textbackslash{}Git\textbackslash{}cmd\textbackslash{}) er i din path\footnote{Se evt \url{http://www.computerhope.com/issues/ch000549.htm}}.

\section{Opsætning af PowerShell}\label{gitguide:powershell}
PowerShell skal have rettigheder til at hente informationer som en del af følgende opsætning.
Derfor skal \texttt{ExecutionPolicy} ændres fra default (\texttt{Restricted}) til \texttt{RemoteSigned}.
Start PowerShell (som administrator) og udfør følgende kommando:
\begin{lstlisting}
Set-ExecutionPolicy RemoteSigned
\end{lstlisting}

Naviger herefter til en mappe hvor du gerne vil at Git gemmer posh-git.
Når du har gjort dette kan posh-git hentes og installeres.
Det gøres ved at ved at køre følgende kommandoer i PowerShell:
\begin{lstlisting}
git clone https://github.com/dahlbyk/posh-git.git
cd posh-git
.\install.ps1
\end{lstlisting}
Følg herefter installationens vejledninger.

\section{ConEmu}
\textbf{Con}sole \textbf{Emu}lator er en wrapper til din konsol (uanset hvilken konsol du anvender) der gør din Windows konsol til alt den den skulle have været.
Blandt funktionerne kan nævnes transparency, fullscreen, global hotkeys, tabs og quake mode.
Du finder ConEmu på \url{http://sourceforge.net/projects/conemu/}.

ConEmu skal ikke installeres, men giver blot en \texttt{.7z} med executables.
Når ConEmu er startet trykkes \texttt{Win+Alt+P} for at åbne settings.
Alle indstillingsmuligheder vil ikke blive forklaret her - det er der ganske enkelt for mange muligheder til.

\section{Ingen indtastning af username/password}
Til sidst er her en vejledning i hvordan du slipper for at indtaste email og adgangskode hver gang du laver \texttt{push} eller \texttt{pull}.

Hent den seneste version af \textit{Windows Credential Store for Git}\footnote{\url{http://gitcredentialstore.codeplex.com/releases/view/103679}} og start den.
Det skulle gerne være nok.
Hvis konsollen (som programmet starter) bliver hængende, så skal du i din terminal navigere til den mappe du har gemt programmet i og køre:
\begin{lstlisting}
git-credential-winstore -i "C:\Program Files (x86)\Git\cmd\git.exe"
\end{lstlisting}
Husk at rette \lstinline!C:\Program Files (x86)\Git\cmd\git.exe! til stien hvor git.exe er installeret.
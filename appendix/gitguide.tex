\chapter{Git guide for Windows}
\section{Installation af Git}
Git kan hentes på \url{http://git-scm.com/download/win}.

Under installationen kan man vælge at installere Git Bash som terminal.
Det vil jeg personligt ikke anbefale, da der findes en extension til PowerShell der laver *magi* og gør verden (på Windows) til et bedre sted.
Opsætning af dette er forklaret i \ref{gitguide:powershell} \nameref{gitguide:powershell}.
Man kan selvfølgelig også vælge at benytte en GUI - men så er man selv ude om det :P

Start den terminal du har mest lyst til at lege med (cmd.exe kan til nød anvendes her).
Tjek herefter om git er blevet korrekt installeret og at git er tilføjet til din path ved at køre:
\begin{lstlisting}
git --version
\end{lstlisting}
Terminalen skulle gerne svare med din version af git (fx \texttt{git version 1.8.3.msysgit.0}).
Hvis det ikke sker skal du tjekke at stien til git.exe (fx C:\textbackslash{}Program Files (x86)\textbackslash{}Git\textbackslash{}cmd\textbackslash{}) er i din path\footnote{Se evt \url{http://www.computerhope.com/issues/ch000549.htm}}.

\section{Opsætning af PowerShell}\label{gitguide:powershell}

PowerShell skal have rettigheder til at hente informationer som en del af følgende opsætning.
Derfor skal \texttt{ExecutionPolicy} ændres fra default (\texttt{Restricted}) til \texttt{RemoteSigned}.
Start derfor PowerShell (som administrator) og udfør følgende kommando:
\begin{lstlisting}
Set-ExecutionPolicy RemoteSigned
\end{lstlisting}

Naviger herefter til en mappe hvor git kan få lov at oprette en mappe til dig.
Her hentes lækkerierne til PowerShell ved at køre følgende kommandoer:
\begin{lstlisting}
git clone https://github.com/dahlbyk/posh-git.git
cd posh-git
.\install.ps1
\end{lstlisting}
Følg installationens vejledninger.
Slet herefter det clonede repo.

\section{ConEmu (optional - but you want it!)}

\textbf{Con}sole \textbf{Emu}lator er en wrapper til din konsol (uanset hvilken konsol du anvender) der gør din Windows konsol til alt den den skulle have været.
Blandt funktionerne kan nævnes transparency, fullscreen, global hotkeys, tabs og quake mode.
Du finder ConEmu på \url{http://sourceforge.net/projects/conemu/}.

ConEmu skal ikke installeres, men giver blot en \texttt{.7z} med executables.
Når ConEmu er startet trykkes \texttt{Win+Alt+P} for at åbne settings.
Jeg vil ikke forsøge at forklare hvor alting er at finde - det er der alt for mange muligheder til.
Følgende er blot opsætning af default terminal:
\begin{enumerate}
\item Vælg Startup -> Tasks
\item Klik på \textbf{+} i bunden af vinduet
\item Skriv et navn på din task (fx PowerShell - Git)
\item Indsæt følgende i feltet i bunden:
\begin{lstlisting}
%SystemRoot%\syswow64\WindowsPowerShell\v1.0\powershell.exe -cur_console:a
-cur_console:d:C:\Users\Mikkel\Documents\Git
\end{lstlisting}
Sidste del (\lstinline!C:\Users\Mikkel\Documents\Git!) angiver den mappe PowerShell skal startes i.
\end{enumerate}

Der findes en masse andre muligheder ift opsætning af ConEmu - bare rod lidt rundt i settings ;)

\section{Ingen indtastning af username/password}
Til sidst er her en vejledning i hvordan du slipper for at indtaste email og adgangskode hver gang du laver @push@ eller @pull@.
Et alternativ er naturligvis at anvende ssh - men det kan være problematisk på Windows.

Hent den seneste version af \textit{Windows Credential Store for Git}\footnote{\url{http://gitcredentialstore.codeplex.com/releases/view/103679}} og start den.
Det skulle gerne være nok.
Hvis konsollen (som programmet starter) bliver hængende, så skal du i din terminal navigere til den mappe du har gemt programmet i og køre:
\begin{lstlisting}
git-credential-winstore -i "C:\Program Files (x86)\Git\cmd\git.exe"
\end{lstlisting}
Husk at rette \lstinline!C:\Program Files (x86)\Git\cmd\git.exe! til stien hvor git.exe er installeret.
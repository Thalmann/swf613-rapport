\label{app:interview-2014-04-03}\textbf{Summary written by the contact person: Mette Thomsen.}
\\
\textit{For birken er det nok for dem at bilen viser at stemmen er høj eller lav. De kunne godt lide at sætte et lyd barometer ind i spillet så børnene fx kan ses hvor højt de snakker på en skala fra 0-10. De vil gerne have et ikon der slår dette fra eller til. Det må gerne være muligt at stoppe bilen. De laver så selv koblingen med børnene mellem spil og virkeligheden. En ide de havde, var at forhindringerne var vurderet på denne skala, og at bilen så også havde et nummer der opdateret alt efter hvor bilen var. Opdeling så midten er 4-6 og alt andet er over eller under. De vil stadig gerne have mulighed for at stoppe spillet og vise lyd barometeret. De vil gerne kunne indstille antal forhindringer og hastighed. Hvis der er mange forhindringer skal bilen være mindre så der kan navigeres rundt. Lav det så simpelt som muligt! Ikke for mange detaljer på bilen eller andet da borgerne har problemer med de bliver optaget af detaljer. De øver både at borgerne skal snakke lavere og højere. Det må gerne kunne indstilles hvor forhindringerne er så et barn der snakker for højt er forhindringerne deroppe hvor bilen kører når man taler højt. På den måde bliver de tvunget til at ændre sig. De vil meget gerne kunne indstille om det handler om at undgå eller samle ting, og eventuelt begge dele. Men det skal kunne deles op. Hold alt enkelt og simpelt!}
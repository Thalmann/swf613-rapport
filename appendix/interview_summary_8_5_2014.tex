\label{interview_tove_8_5}
Tove had used the application a couple of times since the last sprint review, to see how it would perform and what could be improved.
Tove was satisfied with the settings, but the speed settings could be improved so that the number would make more sense.
Tove had forgotten how to calibrate the microphone, but remembered when showed.
Tove did find it that the children had to use too much power to steer the car, when trying to avoid the obstacles.
She would like the car control too be more sensitive.


The current application is presented.
She is impressed with the new speed settings and the way the speed barometer is used to show that visually.
But she finds the pictographs a little bit too small.
The calibration setting was not changed and is still confusing.
Tove finds that the map editor is fine.
The game gets started and Tove is impressed with the improvement of the control of the car.
Tove plays around with the speed and obstacles and is satisfied with it.
But she finds that the game gets difficult too fast when adding more obstacles.


The input volume from the player should not be a long and continuous.
The input volume should be short iterations of syllables like ''da-da-da'' or ''ba-ba-ba''.
Tove finds the game is hard with more than two obstacles.


The new idea, with picking up objects instead of avoiding them is presented and Tove likes the idea.
Tove also agrees to remove the garages, because the teacher should be in control when making the map.
Instead of the garages their should be a finish line.


The application should have sounds to strengthen the citizens auditory sense.
A sound when picking up an object.
A car sound when the game starts.
A voice which says the button text.
When the player wins a voice should say ''godt gået''.

When missing an object and going through to the finish line, the player should get another chance but not win.
The button text ''fortsæt'' should be changed to ''ny tur''.
The button text ''Menu'' should be changed to ''Færdig''.
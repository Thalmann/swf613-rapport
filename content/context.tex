%intro

\section{The Multi-project}

\subsection{GIRAF}
\textit{Graphical Interface Resources for Autistic Folk} (GIRAF) is a semester-wide multi-project collaboration between all Software 6th semester (SW6) groups.
The project started with the SW6 students of 2011 and has run every year since then, this being the 4th year.

The overall idea of the project is to create a multi-purpose Android application, which can be used to ease the lives of citizens and institutional guardians.
The main goal is to recreate already existing physical tools in digitalized versions.
Additionally, in order to make it an all-use-and-purposes application it should also serve as an administration tool to each individual institution that will put it to use.

\subsubsection{Collaboration with Aalborg Municipality}
GIRAF is a collaboration project with Aalborg Municipality.
This year there is a total of 6 stakeholders (all institutional guardians or otherwise linked to institutions) who will be available during the course of the project.

\subsection{Stemmespillet (Cars)}
Throughout the entire course of this project, the project-group has worked with an application called Stemmespillet (Translates to 'The Voice Game' and was previously called 'Cars').
Stemmespillet is a sub-application of the GIRAF-application, with its own purpose, that of assisting citizens in learning to control their pitch of speech.

\subsubsection{Collaboration with stakeholders}
In order to establish its overall purpose and to make it as useful as possible, there were three main stakeholders that were cooperated with during the project.
Two are institutional guardians at 'Birken', a home for psychologically disabled children and the third is a speech therapist, associated with Aalborg Municipality.
These stakeholders were the ones with the most to gain from this sub-application, based on the citizens they worked with on a daily basis.
These were also the stakeholders with the most input, both during interviews and sprint reviews.

\subsubsection{Collaboration with multi-project}
Stemmespillet is a sub-application of the GIRAF-application and therefore needed to follow certain project-standards.
For Stemmespillet there were two overall standards; shared data and shared GUI.
The shared data was for two purposes; re-acquiring saved data in case of replacement of the tablet and for multi-user tablets.
Following GUI standards was to give the application as a whole the same look and feel.

\section{Organization}

\subsection{Method}

\subsubsection{Sprints}

\subsubsection{Meetings}

\subsubsection{Specialists}

\subsection{Stakeholders}

\subsection{Shared Components}

\subsection{Tools}

\subsubsection{Redmine}

\subsubsection{Git}

\subsubsection{Jenkins}
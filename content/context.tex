\section{GIRAF}

\textit{Graphical Interface Resources for Autistic Folk} (GIRAF) is a semester-wide multi-project collaboration between all Software 6th semester (SW6) groups at Aalborg University.
The project started with the SW6 students of 2011 and has run every year since then (this being the 4th year).

The overall idea of the project is to create a multi-purpose Android application, which can be used to ease the lives of citizens and their institutional guardians.
The main goal is to recreate already existing physical tools/methods in digitalized versions.
Additionally, in order to make it an all-use-and-purposes application, it will also serve as an administration tool for each individual institution that will put it to use.

GIRAF is run in collaboration with Aalborg Municipality.
This year there is a total of 6 stakeholders (all institutional guardians or otherwise linked to institutions) who will be available during the course of the project.

\section{Organization}
The multi-project consists of 60 people split into 16 project-groups, each group consisting of 4 people (except for two groups, that was the split of a former 4-man group).

\subsection{Method}
Due to the high level of collaboration and the focus on getting out a usable and working product, an agile methodology was in mind when planning the process.
The method chosen was Scrum of Scrums.
Each project-group would work as a team, with their own day-to-day activities, focusing on their current project during the sprint.
Then there would be common weekly status meetings during the sprints, each sprint concluded with sprint-review and sprint-planning meetings.

\subsubsection{Sprints}
The semester was split into 4 sprints, with the last three sprints being of same length (The first was a bit shorter, as this was meant as a sprint solely for getting to know the former projects).
In order to take into consideration the courses that were running simultaneously with the multi-project, sprint lengths were based on available hours (with the subtraction of course hours).

At the end of each sprint, a sprint review meeting would be held, where all stakeholders were invited.
Here the current products (if possible) would be presented by each group and critiqued by stakeholders and the other groups.
Also, the product backlog would be presented for the stakeholders, in order to get some input on what to focus on in the following sprint.

Following the sprint review, and without the stakeholders, the next sprint would be planned.
Here groups had a chance to switch focus, if their current project was finished, or if something more important had presented itself as a result of the sprint review.

\subsubsection{Meetings}
In order to stay on top of potential problems, a status meeting was held every week (with exception of sprint-end weeks).
The meetings started out with a status report, where a representative from each group presented what they were currently working with, along with any potential problems that needed attention.

If there were problems, these were noted and the proper people informed, so that they could discuss these, without everyone needing to be part of the discussion.
This saved a lot of time, as opposed to having everyone discussing every problem.

However, subjects that did need to be discussed with all groups present would be added to the meeting agenda and then discussed at the meeting.

\subsubsection{Specialists}
In order for people to concentrate on their respective projects, a few people were named specialists in different areas (such as server, contact with stakeholders, and version control).
Each person were responsible for their area; that it was working as it should, as well as helping people with possible problems.

\subsection{Stakeholders}
Day-to-day contact with stakeholders went through a specialist (contact person).
As mentioned, all stakeholders were invited to each sprint review, where they would give feedback to the presented work as well as available for questions.
If any more contact was needed, such as arranging a meeting, this was conveyed by the contact person.

\subsection{Shared Components}\label{multiproject:sharedcomponents}
There were two main shared components; database and GUI components.

The database consisted of the storage of shared data, as well as providing an interface for accessing this data.

The GUI components consisted of shared graphical components, such as backgrounds, buttons and other things that were commonly used and needed to look the same across applications.

As an addition to the shared components, whenever needed, groups would collaborate on minor shared components that could be used across applications.

\subsection{Tools}
In the beginning of the project two representatives from the previous semester told about the tools they used on their semester and their experience with them.
Based on these experiences a couple of tools were chosen to aid the project.
These tools will be described in the following sections.

\subsubsection{Redmine}
Redmine was used as a center of information, consisting of three main uses; forums, wiki-pages, and issue-trackers.

The forums were used for non-pressing matters.
Here, updates to shared components were posted, along with possible instructions if changes needed to be applied to concerned projects.

Wiki-pages were used for static information.
There was a main wiki-page, containing group contact informations, general guides, and meeting summaries.
Additionally, each sub-project had the possibility for its own wiki-page, to contain whatever information about their project that needed to be shared with others.

Issue-tracking was a tool for tracking progress of a project, as well as the possibility to state issues for other projects that ones own project needed done.
Issue-tracking also made it easier for a group to take over the projects if they changed focus after a sprint.

\subsubsection{Git}
Git was used as version control.
Since one of the group members was the Git-specialist, this is the focus of one of the collaboration chapters (see \ref{collaboration:git}).
\bruno{Thomas: Man kunne evt. svare på følgende spørgsmål: - Hvad er fordelene med version kontrol?
- Er der nogle specielle grunde til at git blev foretrukket over andre muligheder?}
\mikael{Don't know what more to write here if anything.}

\subsubsection{Jenkins}
\mikael{Don't know what to exactly write here}.

\section{Stemmespillet (Cars)}
Throughout the entire course of this project, the project-group has worked with an application called Stemmespillet (Translates to 'The Voice Game' and was previously called 'Cars').
Stemmespillet is a sub-application of the GIRAF-application, with its own purpose; that of assisting citizens in learning to control their pitch of speech.

\subsection{Collaboration with stakeholders}
In order to establish its overall purpose and to make it as useful as possible, there were three main stakeholders that were cooperated with during the project.
Two are institutional guardians at 'Birken', a home for psychologically disabled children, and the third is a speech therapist, associated with Aalborg Municipality.
These stakeholders were the ones with the most to gain from this sub-application, based on the citizens they worked with on a daily basis.
These were also the stakeholders with the most relevant input, both during interviews and sprint reviews.

\subsection{Collaboration with multi-project}
Stemmespillet is a sub-application of the GIRAF-application and therefore needed to follow certain project-standards.
For Stemmespillet there were two overall standards; shared data and shared GUI.

The shared data was for two purposes; re-acquiring saved data in case of replacement of the tablet, and for multi-user tablets.
Following GUI standards was to give the application as a whole the same look and feel.

The project group also had two specialists; Git and Redmine Wiki.
The first one took up a lot of time for the concerned person, which also led to the collaboration chapter on the subject of Git.
The latter did not take up a lot of time.

\mikael{Der skal skrives et underafsnit om vores metode i gruppen, SCRUM skal nævnes}
\chapter{Git}\label{collaboration:git}
\newcommand{\git}{Git}\newcommand{\Git}{Git}
\newcommand{\gitolite}{Gitolite}\newcommand{\Gitolite}{Gitolite}
\begin{center}
\textit{The following chapter regards \git{}, the server system \gitolite{} and the use of the two.
Thus a basic knowledge of \git{} is expected on behalf of the reader.}
\end{center}
The code-base for the various applications and sub-projects, that are part of the GIRAF project, is quite big and requires that multiple people (possibly from different groups) can work on the same project at once.
Thus a content management system is required to manage the code and its various revisions. 
It was unanimously decided by all project groups to use \git{} as this content management system.
One of the reasons for using \git{} was that that an administration tool had already been installed on the GIRAF server.
In \cref{git:gitolite} this tool is described along with a guide to its usage.

Some of the students partaking in the project were new to the use of \git{}.
Because of this a \textit{''git specialist''} was chosen.
This specialist was a member of the group developing \textit{Cars}.

\section{\Gitolite{}}\label{git:gitolite}
\Gitolite{} is an administration tool for git repositories.
Interaction with \gitolite{} is done through a git repository in which all repos are defined in one or more configuration files.
Using hooks, \gitolite{} will create repositories when new ones are defined.
The syntax of these configuration files is described in \cref{git:gitolite:config}.

\paragraph{Accessing the server}
Each GIRAF repository is exposed through two urls.
One is read-only and is publicly available.
This allows projects to have other repositories as submodules, but without having write-access.
It also simplifies the server's automated build, as no authentication is required.
The second url is one, to which access is restricted (see \cref{git:gitolite:config}).
The two urls are as follows:
\begin{center}
\url{http://cs-cust06-int.cs.aau.dk/git-ro/} (read-only)\\
\url{http://cs-cust06-int.cs.aau.dk/git/} (read-write)
\end{center}
Each url lists a collection of repositories.
The urls for a repo is equal to appending the repo name to either of the above urls.
Connecting to these repositories can be done using LDAP authentication.
This means that the standard @student.aau.dk logins used for AAUs systems will apply to \gitolite{}.

\subsection{Configuration files}\label{git:gitolite:config}
As mentioned above, repositories in \gitolite{} are configured through a \git{} repository.
This repository can be found at the url below:
\begin{center}
\url{http://cs-cust06-int.cs.aau.dk/git/gitolite-admin/}
\end{center}
In the \texttt{conf} directory in the repository, a collection of files define all the GIRAF repositories.
To each repository a set of access rules apply.
These are described throughout the following section.
Note, however, that the access rules described below are only those applied in the course of the 2014 spring semester.
\Gitolite{} has an online manual\footnote{\url{http://gitolite.com/gitolite/master-toc.html}} for its usage to which one should refer for any additional information.

\subsubsection{Users and groups}
In addition to defining repositories, \gitolite{} allows the definition of groups.
A group can be either a group of repositories or a group of users.
In the context of GIRAF only the latter is applied.
In the two files \texttt{sw6f13-groups.conf} and \texttt{sw6f14-groups.conf} the project groups for each semester are defined.
This scheme could be repeated for other semesters.
A group is defined by prefixing a name with a \textit{@} symbol, as below:
\begin{center}
\texttt{@sw600f14 = user1@student.aau.dk user2@student.aau.dk}
\end{center}
This creates a group called \texttt{@sw600f14} with two members.
The use of AAU email addresses allows the usernames to correspond directly to the LDAP authenticated users.
A group can also consist of other groups, such that a group can be created from the list of all semester groups.

\subsubsection{Multiple configuration files}
The \texttt{.conf} file loaded by \gitolite{} is the \texttt{gitolite.conf}.
As the list of repositories grow, multiple configuration files can be created and included.
To include a configuration file \textit{repos.conf} \gitolite{} has an \texttt{include} command:
\begin{center}
\texttt{include "repos.conf"}
\end{center}
This command can also be used in configuration files included using the command, and is useful when trying to separate various types of repositories.

\subsubsection{Defining a repository}
Defining a new repository in \gitolite{} is very simple.
It must simply be defined, staged, commited and pushed (as any other content in git).
When pushed to the server \gitolite{} will create the new repository.
The following command defines a repository name newrepo:
\begin{center}
\texttt{repo newrepo}
\end{center}
Note that this definition does not explicitly tell \gitolite{} to create the new repository, just that it should exist.
Thus, the same repo can be defined multiple places with different sets of access rules.
This allows for separating different types of access rules into different files (for instance a separate config file can hold any temporary access rules).

\subsubsection{Access rules}
Below in \cref{git:gitolite:example} is the definition of a fictitious repository \texttt{component}.
Using this as an example, the various types of access rules will be described.
There are four\footnote{There are additional advanced access types, but these have not been used for the GIRAF project} types of access available in \gitolite{}.
The terms \textit{push} and \textit{write} are used interchangeably, as is \textit{pull}, \textit{fetch} and \textit{read}.
The four access types are:
\begin{description}
\item[R] Allows the users listed to read from the repository, but not write.
\item[RW] Allows the users listed to read from and write to the repository.
\item[RW+] Similar to \textbf{RW}, but also allows deletion.
This access type allows removal of remote branches and forced push.
In the context of GIRAF, this type of access has been reserved for the git specialist.
\item[- (dash)] The users listed are not allowed to read from or write to the repository.
This can be used to cancel parts of broadly defined access rights.
\end{description}
\begin{lstlisting}[caption={A \gitolite{} configuration file},label=git:gitolite:example]
repo component
    R   = @all
    RW  = username1 username2
    RW+ = @admin
    RW VREF/NAME/dir1/ = username1
    RW VREF/NAME/dir2/file = username1
    -  VREF/NAME/ = username1
    RW branchname = @fixers
\end{lstlisting}
In \cref{git:gitolite:example} we see examples of all the access types.
Below is a list of how each of the lines in the example should be read.
\begin{itemize}
\item Everybody can read from the repository.
The \texttt{@all} usergroup is predefined by \gitolite{} to be all users or all repositories, depending on the context.
\item \texttt{username1} and \texttt{username2} have read and write access to the repository.
\item Administrators can delete information from the repository.
The \texttt{@admin} is not predefined by \gitolite{}.
\item \texttt{username1} has read and write access to the directory \texttt{/dir1/} and the file \texttt{/dir2/file}, but does not have access to any other files.
Notice that line 3 is required for \texttt{username1} to be able to write to the repository at all.
The \texttt{VREF} rule is checked when the actual push is happening.
There are additional rules available to \gitolite{} such as only allowing users to push a certain number of times or only at specific times of day.
\item The \texttt{@fixers} usergroup can read and write to the \texttt{branchname} branch in the repo.
\end{itemize}
Note that the first rule specifies that everyone can read from the repository.
This in turn means that the remaining rules only restrict write access to the repository.

\section{Common \git{} difficulties}
\Git{} is a quite flexible tool for managing code.
One of the reasons for this is the number of commands provided by \git{}.
This was the main cause of problems with the usage of \git{}, as many of the student working on the GIRAF project were new to \git{}.
Thus the git specialist spent a lot of time supporting other groups in the efforts to use \git{}.
This included, in part, an introductory presentation to \git{} early in the semester.
The following chapter will describe some of the typical issues with \git{} encountered during the project.

\subsection{Staging content}
As most users had previously only been exposed to the SVN content management system, there were some confusion as to \git{}s additional \textit{staging} and \textit{pushing} steps in the commit process.
Where SVN copies all changes to files from your local machine to a server in one step, \git{} allows local staging of content, local storing of commit before eventually pushing to the server.

This difference was not properly understood by many of the groups, as they continued in the SVN workflow of always sharing each change (in order to keep commits small), or only committed seldom yielding large commits that were harder to merge with the remaining group members.

Another problem that arose from this was the use of the \texttt{git add .} command, which stages all content in the current directory.
This command (without the use of \texttt{git status} and \texttt{git diff}) renders \git{} blind to what they are actually committing.
The result was that often undesired code wash pushed to the shared server.
A special type of problem resulting from this misuse of \git{} is explained in \cref{git:sub:sha}.

It is the view of the \git{} specialist that more time should be spent discussing the proper use of \git{}, in order to take advantage of the possibilities of git and avoid many of the problems associated with \git{}.

\subsection{Branching and merging}
The understanding of the difference between a local branch and remote branch was hard to grasp for many students.
Similar to the problems described above, the typical issue was that a true copy of a centralized server was expected - as is the case with SVN.
The concept of what is stored locally and what is stored remotely was not properly explained and/or understood.
The result of this was that many branches intended as local-only were pushed to the server and consequently created by others working on the same project.

Similarly, the idea of \textit{''parallel''} work on multiple local branches seemed a scary thought to many.
Some simply kept to the linear world they were accustomed to.

The confusion that arose from the multiple branches never lead to big problems, but mainly to insecurity due to the complexity of \git{}.
Thus a lot of time was spent by the \git{} specialist in an effort to clarify these difficulties.

\subsection{Submodules}
The greatest cause for difficulty regarding \git{} was the use of submodules.
In \git{} submodules represent the inclusion of one repository inside another.
Using this kind of structure allows the nesting of shared components (such as \texttt{OasisLib} and \texttt{giraf-component}) inside other projects.
\Git{} will then allow its users to work with these components as separate repositories while tracking which version of the repositories are in use.
The benefit of this tracking is that when a new version of these components is created, each project can upgrade to the newest version in its own pace.

Many groups had difficulties understanding how the submodules in \git{} worked.
Asides from having to assist in the adding and removal of submodules, the \git{} specialist encountered two general problems in regards to \git{} submodules.
These problems will be explained in the following two sections.

\subsubsection{Referring local changes}\label{git:sub:sha}
As mentioned above, a \git{} submodule references the currently used version of each of its submodules.
This is done by continuously updating a reference to the SHA of the used submodule commit.
As in the example below:

\begin{quote}
\textbf{Example}\\
Let \textbf{A} and \textbf{B} be repositories, where \textbf{B} is a submodule in \textbf{A}.
\textbf{A} keeps track of which version of \textbf{B} is in use by \textbf{A} by registering the SHA \texttt{2b8c053}\footnote{\Git{} SHAs are 40 character strings, abbreviations are used in the examples.}

When a new commit 551b569 is created in \textbf{B}, \textbf{A} does not automatically use is.
Only when a commit in \textbf{A} updates the SHA of \textbf{B} to 551b569, \textbf{A} will use the new commit.

This allows for content to be added to \textbf{B} independently of \textbf{A}, parallelizing the process.
\end{quote}

As a commit in \git{} is only a local operation, nothing is shared with the remaining project groups before the changes are pushed to a server (or another member of the project group).
If only the changes in \textbf{A} are pushed, then anyone trying to use this newest version of \textbf{A} will not be able to use the correct version of \textbf{B} as it only exists locally on one machine.

This problem has arisen, as members of some groups have committed content in repositories to which they have no write access.
They are then not able to push their changes but only their reference to the content that they can not push.
In particular this problem exists due to the wrongful use of \texttt{git add .} where all content is blindly staged without concern for what content should actually be committed.

\subsubsection{Deep clone}\label{git:sub:deep}
The GIRAF repository \texttt{OasisLib} is referenced by multiple project.
Amongst those are both \texttt{giraf-component} and \texttt{cars}.
As the latter also includes \texttt{giraf-component} this infers a duplicate inclusion of \texttt{OasisLib} in \texttt{cars}.
The situation has been illustrated in \cref{git:sub:clonefig}.
\begin{figure}[h]
\centering
\begin{tikzpicture}[node distance=0.8cm, thick, every node/.style={draw=black,thick}]
  \node (cars) {\texttt{cars}};
  \node[below right = of cars] (oasis1) {\texttt{OasisLib}};
  \node[below = of oasis1] (giraf) {\texttt{giraf-component}};
  \node[below right = of giraf] (oasis2) {\texttt{OasisLib}};
  \draw[->] (cars) |- (oasis1);
  \draw[->] (cars) |- (giraf);
  \draw[->] (giraf) |- (oasis2);
\end{tikzpicture}
\caption{Deep clone of the \texttt{cars} repository.}
\label{git:sub:clonefig}
\end{figure}
The figure illustrates the submodules associated with the \texttt{cars} repository (there were however other repositories with a much larger tree of submodules).
In the development environment only the topmost level of submodules were used and thus the remaining were not to be cloned.
However many groups performed a \textit{deep clone} in which they cloned the entire structure.
This is not directly a problem, but the cause of much confusion.
Combined with the problems described in \cref{git:sub:sha} this caused quite a few problems for the groups with many submodules associated with their project.
These problems were all attended to by the git specialist.

\mikkel{Jeg har skrevet følgende i noter til mig selv:\\
Appendix: info on setup and links to git-intro video'er\\
Skal vi have noget i appendix?
}
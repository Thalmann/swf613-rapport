\chapter{Git}
The code-base for the various applications and sub-projects that are part of the GIRAF project is quite big and requires that multiple people (possibly from different groups) can work on the same project at once.
Thus a content management system is required to manage the code and its various revisions. 
It was unanimously decided by all project groups to use Git as this content management system.
One of the reasons for using Git was that that an administration tool had already been installed on the GIRAF server.
In \cref{git:gitolite} this tool is described along with a guide to its usage.

Some of the students partaking in the project where new to the use of Git.
Because of this a \textit{''git specialist''} was chosen.
This specialist was a member of the group developing \textit{Stemmespillet}.
\mikkel{Hvor meget skal der skrives om hvor lang tid der er brugt på support?}

\section{Gitolite}\label{git:gitolite}
Gitolite is an administration tool for git repositories.
Interaction with Gitolite is done through a git repository (\texttt{gitolite-admin}) in which all repositories are defined in one or more configuration files.
Using hooks Gitolite will create repositories when new ones are defined.
The syntax of these configuration files is described in \cref{git:gitolite:config}.

\paragraph{Accessing the server}
Each GIRAF repository is exposed through two urls.
One is read-only and is publicly available.
This allows projects to have other repo's as submodules without having write-access.
It also simplifies the servers automated build as no authentication is required.
The second url is one to which access is restricted (see \cref{git:gitolite:config}).
The two urls are as follows:
\begin{center}
\url{http://cs-cust06-int.cs.aau.dk/git-ro/} (read-only)\\
\url{http://cs-cust06-int.cs.aau.dk/git/} (read-write)
\end{center}
Each url lists a collection of repositories.
The urls for a repo is equal to appending the repo name to either of the above urls.
Connecting to these repositories can be done using LDAP authentication.
This means that the standard @student.aau.dk logins used for AAUs systems will apply to Gitolite.

\subsection{Configuration files}\label{git:gitolite:config}

%Gitolite
%Common problems
% - Branches
% - Submodules
%Appendix info on setup and links to git-intro video'er
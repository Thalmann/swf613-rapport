When we started the semester, we had a big planning meeting with all students present.
Here we decided the overall method and structure of the project.
We based our decisions on the experiences of those from the previous year (who in turn had also used the experiences of the year before that).

However, we did still have some things that could have been handled better, or remained problematic but without a better solution.
We will present these problems, and possible solutions, so that next year's multi-project students can use our experiences.

\paragraph{Early meetings with stakeholders} is something we wish we had, but did not.
We had our first meeting at the end of the second sprint and our second meeting in the beginning of the 4th sprint.
This resulted in us establishing requirements way too late, and having to do major changes throughout the entire semester, as requirements changed drastically as a result of meetings with the stakeholders.

It would instead be a good idea to get a good overview of the project(s) and early on clarify any confusions, by meeting with the stakeholders.

\paragraph{Plan sprint-reviews early} in the semester.
Our first sprint-review had only 2 stakeholders present, and they had to leave early.
This was due to them being invited only shortly before the sprint-review.

It would be a good idea to establish sprint-reviews early on and invite stakeholders as soon as these are planned.
Additionally, depending on response, it could be a good idea to move sprint-reviews to fit the needs of the stakeholders, as it is really important that they can make the meeting, rather than every single person from the project-groups.

\paragraph{Requirements} was something that was determined during the first sprint by 2 groups, however, these were mostly too general and therefore not very useful.
Also, these requirements were not maintained or updated throughout the semester.
Instead, informal rules formed as a result of status meetings, individual group's meetings with stakeholders, and from sprint-reviews.

It would be a good idea to establish concrete requirements early on, and then update these as new information is gained throughout the project.

\paragraph{Status meetings} need to be short and to the point.
At first our status meetings were long and, for some, pointless.
This was due to everything being discussed by all present, even though some subjects only concerned some groups/individuals.

In the end the meetings worked a lot better, as they were very short, consisting of a short status report from each group, and if there were any problems, the concerned parties would meet or plan a meeting after the status meeting.

\paragraph{Working independently from GIRAF}
As cars is a game and does not use the database containing pictograms we were not depending as much on the overall project as most of the other groups.
This made it easier to progress without disturbances, and this could also be seen at the sprint reviews as we were one of the few groups that had something to show at all four sprint reviews.
On the other side it made it harder to have an influence on the project, because we did not have a direct dependence on the matters that were discussed on the status meetings.

\paragraph{Redmine Forums} was meant as a source of discussions on non-pressing matters.
It did however not work well for this.

Firstly, the forum itself works poorly.
The layout is not very good, and its functionality very limited.
A better way of viewing what is new since last time you checked is dearly needed.

Secondly, for it to work, people need to check the forums regularly, which far from everyone did.

Another option would be to activate notifications by email.

\paragraph{Issue-tracking} is something that never quite had the effect that we wished.
The intention was that everyone had their issue-tracker updated, so that it was possible to see what each group was currently working with.
This is hard due to the different levels of details, and in some cases complete absence, of the issues.

Another problem, which we had, is that we also like to use a real-life scrum-board, with post-its as issues.
This made it necessary to have both Redmine Issues as well as real-life tasks synchronized, which was a tedious task.

\paragraph{Report guidelines} were a confusing matter throughout the entire semester.
One thing we should have considered from the start, is the fact that they are only guidelines, not rules.

We thought the report structure described in the guidelines as confusing and far from what we were used to, so we had a hard time adapting to this.

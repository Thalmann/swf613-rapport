This section contains an overview of the requirements and whether they have been fulfilled during this sprint.
If they are fulfilled it is explained how.

\begin{tabularenumerate}
\begin{table}[H]
\begin{tabular}{c|l|c|c}
\textbf{\#} & \textbf{Requirement} & \textbf{Solved} & \textbf{Link} \\
\hline
\tabenum \label{sprint1:tab2:req1} & \begin{tabular}[l]{@{}l@{}}The game must not be a side-scrolling game,\\because the citizen must be able to see the goal\end{tabular}
 & $\surd$ & \cref{sprint1:req1} \\
\hline
\tabenum \label{sprint1:tab2:req2} & \begin{tabular}[l]{@{}l@{}}The game must make it possible\\for the citizen to control a car\end{tabular}& $\surd$ & \cref{sprint1:req2} \\
\hline
\tabenum & The game should contain stars as points & $\times$ & - \\
\hline
\tabenum \label{sprint1:tab2:req4} & The goal of the game is to reach a garage & $\surd$ & \cref{sprint1:req4} \\
\hline
\tabenum & When the game is won an reward should be given & $\times$ & - \\
\hline
\tabenum & \begin{tabular}[l]{@{}l@{}}It must be possible\\to save and load settings for a specific profile\end{tabular} & $\times$ & - \\
\hline
\tabenum & It must be possible to calibrate the microphone & $\times$ & - \\
\hline
\tabenum & \begin{tabular}[l]{@{}l@{}}It must be possible\\for the user to change the difficulty of the game\end{tabular} & $\times$ & - \\
\hline
\end{tabular}
\caption{Requirements fulfilled after sprint 1.}
\label{sprint:requirement_table_2}
\end{table}
\end{tabularenumerate}

\paragraph{\Cref{sprint1:tab2:req1}}\label{sprint1:req1}
Looking at \cref{product-sprint1} the game starts at the left side of the screen and ends at the right side of the screen, without any scrolling.
\paragraph{\Cref{sprint1:tab2:req2}}\label{sprint1:req2}
It is possible to control the car with volume, making the car move up when the input volume is high and down when it is low.
The car stays at its position when there is no input.
For more detail look \cref{sprint1:product}.
\paragraph{\Cref{sprint1:tab2:req4}}\label{sprint1:req4}
Looking at \cref{product-sprint1} on the right side there are three boxes labelled 'garage', these represent the place where the graphics for the garages should be.
How they work is explained in \cref{sprint1:product}.
It seems that the authors of the code are not familiar with the object oriented programming concept inheritance.
Inheritance in object oriented programming is when you have a class based on an other class.
This prevents code duplication which improves maintainability

An example of good use of inheritance is if we have a class called \textit{Vehicle}.
Which e.g. has a method \textit{ChangeOwner} and a field \textit{RegistrationNumber}.
We can then make a specialisation of \textit{Vehicle} e.g. called \textit{Car} and \textit{Truck}.
These two new classes can now also use the method and field there is in \textit{Vehicle}.
This means the code gets reused, instead of having it in both \textit{Car} and \textit{Truck} we have it only in \textit{Vehicle}.

An example from 'Cars' is the class \lstinline!GameObject! which is an empty super class to the following classes:

\begin{tabular}{ c  c }
\parbox{\textwidth/2}{
\begin{itemize}
\item \lstinline!Barricade!
\item \lstinline!Bump!
\item \lstinline!Car!
\end{itemize}} &
\parbox{\textwidth/2}{
\begin{itemize}
\item \lstinline!Cat!
\item \lstinline!Garage!
\item \lstinline!Rock!
\end{itemize}
}
\end{tabular}
The classes shown above all have a \textit{draw} method which are identical.
This is a classic example on a method which should have been in the super class.
If it would have been in the super class all its derived classes could have used the method.
This would then be good code reuse.
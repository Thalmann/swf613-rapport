\label{sprint1_inheritance}
It seems that the object oriented programming concept inheritance is not used in the project 'Cars' code.
Inheritance in object oriented programming is when you have a class based on an other class.
This prevents code duplication which improves maintainability.

An example from 'Cars' is the class \lstinline!GameObject! which is an empty super class to the following classes:

\begin{tabular}{ c  c }
\parbox{\textwidth/2}{
\begin{itemize}
\item \lstinline!Barricade!
\item \lstinline!Bump!
\item \lstinline!Car!
\end{itemize}} &
\parbox{\textwidth/2}{
\begin{itemize}
\item \lstinline!Cat!
\item \lstinline!Garage!
\item \lstinline!Rock!
\end{itemize}
}
\end{tabular}
The classes shown above all have a \textit{draw} method which are identical.
This is a classic example on a method which should have been in the super class.
If it would have been in the super class all its derived classes could have used the method.
This would then be good code reuse.
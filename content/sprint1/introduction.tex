\mikael{This introduction was written for when we had refactoring as a goal. It needs to be rewritten so that it fits the new goal.}
In this sprint we focus on the 'Cars' project from last year's multi-project.
The main purpose of the sprint is to get an understanding of the project, while making it usable by eliminating bugs and implementing missing features.

\paragraph{Cars} is a small game, in which the user controls a car, automatically moving from left to right on a three-lane road.
The user must use voice input, based on pitch, to move the car up and down in order to dodge obstacles on the road and in the end manoeuvre the car into a garage.
A screenshot from the 'Cars' app can be seen in \cref{fig:cars_screenshot}.

\begin{figure}[h]
\centering
\includegraphics[width=.5\textwidth]{sprint1/cars_screenshot}
\caption{Screenshot from the 'Cars' app.}
\label{fig:cars_screenshot}
\end{figure}

\paragraph{Sound input} is an important element of 'Cars', as it is used to control the car.
It is also a part of the game which as of now does not work properly.
The car's movements do not correctly correspond to the voice input, seemingly causing the car to move at random, no matter what kind of pitch is used.

Using voice input based on pitch was an original requirement of the 'Cars' project, worked out between the Cars group and the stakeholders.
Its intention was to solve a particular problem concerning some autists, not being able to control their pitch during speech.

Based on its importance and that it does not work, this is a natural major focus point during the first sprint.
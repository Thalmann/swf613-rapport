\subsection{Suggestion}
To best avoid ending in a situation similar to the above, it was deemed important that a framework was found on top of which the new game could be built.
Such a framework could define a \textit{proper} structure for game development, allowing the project group to focus solely on the actual game development

\paragraph{Kilobolt}\label{kilobolt:description}
In \textit{''Beginning Android Games''}\cite{androidgames} an open source framework is suggested for game development.
The framework is presented and described on kilobolt.com\cite{kilobolt} using a simple example of a game.
The framework generalizes Java game development by defining a set of interfaces describing the structure of a game.

On top of these interfaces an Android version is implemented specifying how the generalized framework should be executed on Android devices.
In the implementations presented on kilobolt.com there are some inconsistencies between the interfaces and Android implementations.
An example of this is that in some cases the Android implementation defines more methods than the interface that is implemented.
Any such inconsistencies would have to be fixed if implementing for multiple platforms.
Simultaneously any future expansion of the framework would require all methods to be defined in both interfaces and Android implementations. 

Since the project for which the framework is used is Android-only, the generalized part of the framework was removed.

Understanding a method is difficult when it is big and has different unrelated operations.
It would improve the readability of the code to split them up to small methods instead, for instance an operation per method.

An example of an operation could be to fill a whole array with a certain number.
The code for this would be a for-loop filling an array with the number.
Instead of having the code in the method, where you then will use it, it is better to extract it to a new method and call that.
This also improves readability by providing a describing name for the method making it easier to tell the intention of the method.
Keeping methods short makes the maintenance and refactoring of the code easier, because the code then have a higher readability.

The \textit{Run} method (\cref{big_run_method}) from 'Cars' from the class \lstinline!RecorderThread! is an example of a method which is impossible to command and should be divided into several sub methods.
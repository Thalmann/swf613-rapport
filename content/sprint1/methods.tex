This paragraph will contain the general problems with the methods in the project 'Cars' and their usability.

Understanding a method is difficult when it is big and has different operations.
One should instead split them up to small methods, for instance an operation per method.

An example of an operation could be to fill a whole array with a certain number.
The code for this would be a for-loop filling an array with the number.
Instead of having the code in the method, where you then will use it, it is better to create a method for that and call it.
This in general makes the maintenance and refactoring of the code easier, because the code then has a higher readability.

The \textit{Run} method(\cref{big_run_method}) from 'Cars' from the class \textit{RecorderThread} is a classic example on a method which should be smaller.
An interview was carried out April 4th 2014 with two pedagogues from Birken; Mette Als Andreasen og Kristine Niss Henriksen.

\subsection{Purpose}
At the start of the sprint we were still solely working out from the requirements found in last year's Cars project report.
However, we found that some of these requirements were still not clear.
Also, the overall purpose and application of the Cars app was not clear either.

The goal of the interview was then to establish the overall purpose of the Cars app; how it would be used in practice and how it would relate to real-life problems.
Another important issue was to figure out how to adjust difficulty in order to make the game more usable, and not be too easy or difficult for some.

\subsection{Planning}
Prior to the interview we agreed upon topics we would like discussed with the stakeholders.
We decided to stick to topics, and not specific questions, to allow a more open discussion.
This was because we didn't want to lock ourselves too much to the old requirements, in case there were some bigger and better changes available.

There were two main topics; overall purpose and use of the app, and how to customize difficulty.
For the difficulty part (this is linked to the old requirement: \textit{It must be possible for the user to change the difficulty of the game}) we would like to discuss variables like speed and size of the car, number and size of obstacles, and number of garages.
Additionally there was mentioned points in the old requirements, we would also like this clarified, as it was not implemented by the previous Car project group.

\subsection{Result}
The results of the interview will be presented here.
A resume of the interview can be seen in \ref{app:interview-2014-04-03}.

\begin{enumerate}
\item The car is controlled in such a way, that the vertical position of the car is relative to the current loudness of the player's voice.
\item There is a digit between 0 and 10 displayed on the car as well as obstacles, representing the loudness level, based on its vertical position. 
\item Besides the scales from 0 to 10, both speed and loudness have pictograms illustrating some of the values on these scales.
\item It should be possible to pause the game. When the game is paused, a loudness-barometer is displayed next to the car, further visualizing the current loudness.
\item Speed is alterable. The speed level is represented as a digit between 0 and 10.
\item The placement and number of obstacles is alterable.
\item The placement of obstacles should be in such a way, that it is possible to adapt it to both citizen with tendency to speaking too loud as well as those speaking too low.
\item The graphics need to be simple, as some citizens have a low attention span and are easily distracted.
\item It should be possible, in settings, to switch between avoiding objects and picking objects up.
\item When picking objects up, this is linked to pictogram categories.
\item It is important that the pickup/category ''mode'' is optional, as not all development levels are capable of both sound mode and category mode.
\end{enumerate}
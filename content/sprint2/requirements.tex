Initially the requirements were the same as last sprint.
However, due to the interview carried out during the sprint, some new requirements were gathered, along with some of the old requirements being changed.

Here will be represented a new and complete list of requirements, as a merge between the initial requirements and the newly obtained.

\textbf{Note:} The requirement \textit{The game should contain stars as points} (see \cref{sprint1:requirements}, \cref{sprint1:requirements:points}) was completely removed, as the focus should be on using the voice, not winning as such.

\begin{enumerate}
\item The game must not be a sidescrolling game, because the citizen must be able to see the goal
\item The goal of the game is to reach a garage
\item When the game is won an reward should be given
\item It must be possible to save and load settings for a specific profile
\item It must be possible to calibrate the microphone 
\item The car is controlled in such a way, that the vertical position of the car is relative to the current loudness of the player's voice.
\item There is a digit between 0 and 10 displayed on the car as well as obstacles, representing the loudness level, based on its vertical position. 
\item Besides the scales from 0 to 10, both speed and loudness have pictograms illustrating some of the values on these scales.
\item It should be possible to pause the game. When the game is paused, a loudness-barometer is displayed next to the car, further visualizing the current loudness.
\item Speed is alterable. The speed level is represented as a digit between 0 and 10.
\item The placement and number of obstacles is alterable.
\item The placement of obstacles should be in such a way, that it is possible to adapt it to both citizen with tendency to speaking too loud as well as those speaking too low.
\item \label{graphicsReq} The graphics need to be simple, as some citizens have a low attention span and are easily distracted.
\item It should be possible, in settings, to switch between avoiding objects and picking objects up.
\item When picking objects up, this is linked to pictogram categories.
\item It is important that the pickup/category ''mode'' is optional, as not all development levels are capable of both sound mode and category mode.
\end{enumerate}
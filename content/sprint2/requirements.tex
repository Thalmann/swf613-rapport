Initially the requirements were the same as last sprint.
However, due to the interview carried out during the sprint, some new requirements were gathered, along with some of the old requirements being changed.

Here will be represented a new and complete list of requirements, as a merge between the initial requirements and the newly obtained.

\textbf{Note:} The following is based on \cref{sprint1:requirement_table_1}.
\Cref{sprint1:tab1:req2}  was vague so it was altered to \cref{sprint2:tab1:req2} on \cref{sprint2:requirement_table_1} so it now is more precise.
\Cref{sprint1:tab1:req3} was completely removed, as the focus should be on using the voice, not winning as such.
\Cref{sprint1:tab1:req8} was vague so it was split into several new requirements regarding difficulty from the interview.

\begin{tabularenumerate}
\begin{figure}[H]
\begin{longtable}{c|l|c|c}
\textbf{\#} & \textbf{Requirement} & \textbf{Solved} & \textbf{Link} \\
\hline
\tabenum & \begin{tabular}[l]{@{}l@{}}The game must not be a side-scrolling game,\\because the citizen must be able to see the goal\end{tabular}
 & $\surd$ & \cref{sprint1:req1} \\
\hline
\tabenum \label{sprint2:tab1:req2} & \begin{tabular}[l]{@{}l@{}}The car is controlled in such a way,\\that the vertical position of the car is relative\\ to the current loudness of the player's voice.\end{tabular}& $\surd$ & \cref{sprint1:req2} \\
\hline
\tabenum & The goal of the game is to reach a garage & $\surd$ & \cref{sprint1:req4} \\
\hline
\tabenum & When the game is won an reward should be given & $\times$ & - \\
\hline
\tabenum & \begin{tabular}[l]{@{}l@{}}It must be possible\\to save and load settings for a specific profile\end{tabular} & $\times$ & - \\
\hline
\tabenum & It must be possible to calibrate the microphone & $\times$ & - \\
\hline
\tabenum & \begin{tabular}[l]{@{}l@{}}There is a digit between 0 and 10\\ displayed on the car as well as obstacles,\\ representing the loudness level,\\ based on its vertical position.\end{tabular} & $\times$ & - \\
\hline
\tabenum & \begin{tabular}[l]{@{}l@{}}Besides the scales from 0 to 10,\\ both speed and loudness have pictograms\\ illustrating some of the values on these scales.\end{tabular} & $\times$ & - \\
\hline
\tabenum & \begin{tabular}[l]{@{}l@{}}It should be possible to pause the game.\\ When the game is paused,\\ a loudness-barometer is displayed next to the car,\\ further visualizing the current loudness.\end{tabular} & $\times$ & - \\
\hline
\tabenum & \begin{tabular}[l]{@{}l@{}}Speed is alterable. The speed level\\ is represented as a digit between 0 and 10.\end{tabular} & $\times$ & - \\
\hline
\tabenum & \begin{tabular}[l]{@{}l@{}}The placement and number of obstacles\\ is alterable.\end{tabular} & $\times$ & - \\
\hline
\tabenum & \begin{tabular}[l]{@{}l@{}}The placement of obstacles should be\\ in such a way,\\ that it is possible to adapt it to both citizen\\ with tendency to speaking too loud\\ as well as those speaking too low.\end{tabular} & $\times$ & - \\
\hline
\tabenum & \begin{tabular}[l]{@{}l@{}}The graphics need to be simple,\\ as some citizens have a low attention span\\ and are easily distracted.\end{tabular} & $\times$ & - \\
\hline
\tabenum & \begin{tabular}[l]{@{}l@{}}It should be possible, in settings, to switch\\ between avoiding objects and picking objects up.\end{tabular} & $\times$ & - \\
\hline
\tabenum & \begin{tabular}[l]{@{}l@{}}When picking objects up, this is\\ linked to pictogram categories.\end{tabular} & $\times$ & - \\
\hline
\tabenum & \begin{tabular}[l]{@{}l@{}}It should be possible, in settings, to switch\\ between avoiding objects and picking objects up.\end{tabular} & $\times$ & - \\
\hline
\tabenum & \begin{tabular}[l]{@{}l@{}}It is important that the pickup/category\\ ''mode'' is optional, due to different capabilities\\ for each citizen.\end{tabular} & $\times$ & - \\
\hline
\end{longtable}
\caption{Requirements fulfilled after sprint 1.}
\label{sprint2:requirement_table_1}
\end{figure}
\end{tabularenumerate}
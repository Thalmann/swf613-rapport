\section{Connection to the database}
\label{sprint3:database}

As mentioned in \cref{multiproject:sharedcomponents}, apps within GIRAF can make use of a shared database, so that profiles and their settings can be saved.

Apps opened within the GIRAF launcher are supplied with two integers; \lstinline|currentChildID| and \lstinline|currentGuardianID|.
The first integer identifies the currently active child, however, this value will be $-1$ if no child is active.
The second integer identifies the current institutional guardian, and is always supplied.

Settings are loaded from the database whenever the settings, map-editor or main activities are started.
Settings are saved whenever the settings or map-editor activities are terminated.

\paragraph{Accessing the database}

In order to access the local database (called LocalDB within GIRAF), a library \lstinline|OasisLib| must be used.
\lstinline|OasisLib| makes available various controllers, the one used for saving and loading settings is called \lstinline|ProfileApplicationController|.
By supplying the current \lstinline|Application|, which is a class that contains all identifying information about an application (name, package-name etc.) and the current profile id, a \lstinline|ProfileApplication| can be obtained, containing the settings for the corresponding application and profile.
\lstinline|ProfileApplicationController| also contains methods for obtaining and modifying the loaded \lstinline|ProfileApplication|.

Settings are saved within \lstinline|ProfileApplication| as a map \lstinline|Setting<String, String, String>|, where the strings represent key, type, and value accordingly.
Cars has 3 setting keys; ''speed'', ''colors'', and ''calibration''.
All values are therefore converted to strings and put in the map with their corresponding key. For colors and calibration, which contain more than one value, different types are also supplied.
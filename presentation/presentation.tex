\documentclass{beamer}
\graphicspath{{../graphics/}}
\usepackage{listings}
\usepackage{ulem}
\usepackage{subcaption}
\captionsetup{compatibility=false}
\usepackage[linesnumbered]{algorithm2e}
\usepackage{multicol}

\newcommand{\linespace}{\vspace{1em}}

\mode<presentation>
{
  \usetheme{Darmstadt}
  \setbeamertemplate{footline}[frame number]
  \setbeamertemplate{navigation symbols}{}
  \setbeamercovered{transparent}
}

\AtBeginSection[]
{
   \begin{frame}
        \frametitle{Indhold}
        \tableofcontents[sectionstyle=show/hide,subsectionstyle=show/show/hide]
   \end{frame}
}

%\usepackage[danish]{babel}
\usepackage[T1]{fontenc}

\usepackage[utf8]{inputenc}

\usepackage{times}

\usepackage{tikz}
\usepackage{multirow}

\title[Mapping med Lego-robot]{Mapping med Lego-robot}

\subtitle{SW505E13}

\author[SW505E13]{Mikkel Sand\o ~Larsen, \and Bruno Thalmann, \and Stefan Marstrand Getreuer Micheelsen, \and Stefan Thilemann, \and Mikael Elki\ae r Christensen, \and Anders R. Nielsen}

\institute[Aalborg University]
{
  Department of Computer Science\\
  Aalborg University}

\date[CFP 2003]{31. Januar 2014}

\begin{document}

%--------------------------------------------------
%     INTRODUKTION
%--------------------------------------------------

\begin{frame}
  \titlepage
\end{frame}

\begin{frame}
    \frametitle{Indhold}
    \tableofcontents[sectionstyle=show/show,subsectionstyle=hide/hide/hide]
\end{frame}

\section{Overblik}

\input{slides/intro}

\section{Løsningsmetoder}
%Løsningsmetoder
\input{slides/losningsmetoder}
%Robottens design
\section{Robottens design}
\input{slides/walle}
\input{slides/teori}
\input{slides/localization}
\input{slides/ruteplan}

\section{Design}
\input{slides/arkitektur}
\input{slides/evaluering}

\end{document}
\documentclass[a4paper,12pt,english]{article}

\usepackage[english]{babel}
\usepackage[utf8]{inputenc}
\usepackage[T1]{fontenc}
% Fixing the font issue
\usepackage{ae,aecompl}

\begin{document}

\section*{GIRAF -- Cars Voice Game}

\begin{description}
\item[Project:] GIRAF -- Cars Voice Game
\item[Group:] SW613F14
\item[Members:] ~ \\
	Bruno Thalmann \\
	Mikael Elki\ae r Christensen \\
	Mikkel Sand\o{} Larsen \\
	Stefan Marstand Getreuter Micheelsen
\end{description}

\subsection*{Resum\'e}

\subsubsection*{GIRAF}
\textit{Graphical Interface Resources for Autistic Folk} (GIRAF) is a semester-wide multi-project collaboration between all Software 6th semester (SW6) groups at Aalborg University.
The project started with the SW6 students of 2011 and has run every year since then (this being the 4th year).

The overall idea of the project is to create a multi-purpose Android application, which can be used to ease the lives of citizens and their institutional guardians.
The main goal is to recreate already existing physical tools/methods in digitalized versions.
Additionally, in order to make it an all-use-and-purposes application, it will also serve as an administration tool for each individual institution that will put it to use.

GIRAF is run in collaboration with Aalborg Municipality.
This year there is a total of 6 stakeholders (all institutional guardians or otherwise linked to institutions) who will be available during the course of the project.

\subsubsection*{Cars Voice Game}
Over the course of the semester, the project-group has developed an application that assists citizens in learning to control the pitch of their voice.
The application is developed the form of a game title \textit{Stemmespillet} (translates to ''The Voice Game'').
In relation to the development this application is also referred to as ''Cars''.

Three of the stakeholders mentioned above were cooperated with during the development of Cars.
This allowed the project-group to better establish the overall purpose of the application and ensure its usefulness to its users.
Of the three stakeholders two are institutional guardians at 'Birken', which is a home for psychologically disabled children.
The third stakeholder is a speech therapist associated with Aalborg Municipality.
When considering the citizens these stakeholders worked with on a daily basis, they were the most relevant stakeholders to the Cars application.

The project group also had two responsibilities; Git and Redmine Wiki.
Each were assigned to one of the group members.
The first one took up a lot of time for the concerned person, which also led to the collaboration chapter on the subject of Git.

To contribute to the multi-project process, the report also contains two chapters to be used by future multi-projects. The first chapter contains a description of the generic framework used to develop the game.
The second chapter is about the usage of the Git version control system in the multi-project, with a detailed explanation of the usage and common mistakes.

\end{document}
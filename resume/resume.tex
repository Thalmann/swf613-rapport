\documentclass[a4paper,12pt,english]{article}

\usepackage[english]{babel}
\usepackage[utf8]{inputenc}
\usepackage[T1]{fontenc}
% Fixing the font issue
\usepackage{ae,aecompl}

\begin{document}

\section*{GIRAF -- Cars Voice Game}

\begin{description}
\item[Project:] GIRAF -- Cars Voice Game
\item[Group:] SW613F14
\item[Members:] ~ \\
	Bruno Thalmann \\
	Mikael Elki\ae r Christensen \\
	Mikkel Sand\o{} Larsen \\
	Stefan Marstand Getreuter Micheelsen
\end{description}

\subsection*{Resum\'e}

\subsubsection*{GIRAF}
\textit{Graphical Interface Resources for Autistic Folk} (GIRAF) is a semester-wide multi-project produced by all Software 6th semester (SW6) groups at Aalborg University.
The project was initiated by SW6 students in 2011 and has been the semester project for SW6 students since.
This is the fourth year the project is running.

The purpose of the project is to create a multipurpose Android application, which can aid autistic children and their guardians.
The goal is to create applications that correspond to the tools and methods they already use.
In order to make the multi-project complete it will also contain administrative tools to 

GIRAF is collaborating with Aalborg Municipality and has 6 stakeholders this year.
These stakeholders are all linked to the institutions and are available for interviewing throughout the project.

\subsubsection*{Cars Voice Game}
Throughout the entire course of this project, the project-group has worked with an application called Stemmespillet (Translates to 'The Voice Game').
In relation to the development this application is also referred to as ''Cars''.
Cars is a sub-application of the GIRAF-application, with its own purpose; that of assisting citizens in learning to control their pitch of speech.

In order to establish its overall purpose and to make it as useful as possible, there were three main stakeholders that were cooperated with during the project.
Two are institutional guardians at 'Birken', a home for psychologically disabled children, and the third is a speech therapist, associated with Aalborg Municipality.
These stakeholders were the most relevant for this sub-application, based on the citizens they worked with on a daily basis.
These were also the stakeholders with the most relevant input, both during interviews and sprint reviews.

The project group also had two responsibilities; Git and Redmine Wiki.
Each were assigned to one of the group members.
The first one took up a lot of time for the concerned person, which also led to the collaboration chapter on the subject of Git.

To contribute to the multi-project process, the report also contains two chapters to be used by future multi-projects. The first chapter contains a description of the generic framework used to develop the game.
The second chapter is about the usage of the Git version control system in the multi-project, with a detailed explanation of the usage and common mistakes.

\end{document}
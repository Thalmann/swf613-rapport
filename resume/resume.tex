\documentclass[a4paper,12pt,english]{article}

\usepackage[english]{babel}
\usepackage[utf8]{inputenc}
\usepackage[T1]{fontenc}
% Fixing the font issue
\usepackage{ae,aecompl}

\begin{document}

\section*{GIRAF -- Cars Voice Game}

\begin{description}
\item[Project:] GIRAF -- Cars Voice Game
\item[Group:] SW613F14
\item[Members:] ~ \\
	Bruno Thalmann \\
	Mikael Elki\ae r Christensen \\
	Mikkel Sand\o{} Larsen \\
	Stefan Marstand Getreuter Micheelsen
\end{description}

\subsection*{Summary}

\subsubsection*{GIRAF}
\textit{Graphical Interface Resources for Autistic Folk} (GIRAF) is a semester-wide multi-project produced by all Software 6th semester (SW6) groups at Aalborg University.
The project was initiated by SW6 students in 2011 and has been the semester project for SW6 students since.
This is the fourth year the project is running.

The purpose of the project is to create a multipurpose Android application, which can aid autistic children and their guardians.
The goal is to create applications that correspond to the tools and methods they already use.
In order to make the multi-project complete it will also contain administrative tools to manage user profiles.

GIRAF is collaborating with Aalborg Municipality and has 6 stakeholders this year.
These stakeholders are all linked to the institutions and are available for interviewing throughout the project.

\subsubsection*{Cars Voice Game}
Over the course of the semester, the project-group has developed an application that assists citizens in learning to control the pitch of their voice.
The application is developed the form of a game titled \textit{Stemmespillet} (translates to ''The Voice Game'').
In relation to the development this application is also referred to as ''Cars''.

Three of the stakeholders mentioned above were cooperated with during the development of Cars.
This allowed the project-group to better establish the overall purpose of the application and ensure its usefulness to its users.
Of the three stakeholders, two are institutional guardians at 'Birken', which is a home for psychologically disabled children.
The third stakeholder is a speech therapist associated with Aalborg Municipality.
When considering the citizens, who these stakeholders worked with on a daily basis, they were the most relevant stakeholders to the Cars application.

\subsubsection*{Responsibilities in multiproject}
The project group developing the Cars application had two responsibilities in relation to the multiproject; Git and Redmine Wiki.
Two members of the group were assigned to these responsibilities on behalf of the entire multiproject.
Though only one per responsibility.
The Git responsibility took up a lot of time in the course of the semester, but was of great benefit to the multiproject.
As such, one of the collaboration chapters in the report is reserved for the description this responsibility.

The second collaboration chapters describes the generic Android framework that was used to develop the Cars application.
This could be of benefit to future project-groups developing games, as part of the GIRAF application.
\end{document}